%%%% DESCRIPTION %%%%%
% Title of Document: 
% Start Date: 
%%%%%%%%%%%%%%%%%%%%%%%%
%%%FINAL CHECK%%%%%%%%%
%read text
%Bibliography list
%spell check
%chklatex
%unusedlabels
%Overfull

\documentclass[12pt, letterpaper, reqno]{amsart}

\usepackage[utf8]{inputenc}
% indicate the language, for example, spanish or russian
\usepackage[english]{babel}
%\usepackage[spanish]{babel}

%comments
\usepackage{comment}

%drawings
\usepackage{graphicx}

%commutative diagrams
\usepackage[all]{xypic}

%%%%%%%% DIMENSIONS %%%%%%%%%%%
\textwidth=16truecm
\textheight=22truecm
\topmargin=0pt
\oddsidemargin=20pt
\evensidemargin=20pt
\baselineskip1.2truecm

%%%%%%%% THEOREMS %%%%%%%%%%%%
\theoremstyle{definition}
%\newtheorem{df}{Definition}
\newtheorem{df}{Definition}
\theoremstyle{plain}
\newtheorem{prop}{Proposition}
%\newtheorem{prop}{Proposición}
\newtheorem{thm}{Theorem}
%\newtheorem{thm}{Teorema}
\newtheorem{lm}{Lemma}
%\newtheorem{lm}{Lema}
\newtheorem{cor}{Corollary}
%\newtheorem{cor}{Corolario}
\theoremstyle{remark}
\newtheorem{ex}{Example}
%\newtheorem{ex}{Ejemplo}
\newtheorem{rem}{Remark}
%\newtheorem{rem}{Observación}

%%%%%%%% COMMENTS (REMOVE THEM IN THE FINAL VERSION!!!) %%%%%%%
\def\comment#1{{\par \tt #1}}

%%%%%%%%% FOR DRAFT%%%%%%%%%%%%%
%\renewcommand{\baselinestretch}{1.5}
%%%%%%%%%%%%%%%%%%%%%%%%%%%

%%%%%%%%%%%% PAPER DATA %%%%%%%%%%%%%%%%%%%
\author{Julián Jiménez Cárdenas}
\address{}


\title[Posgraduate Project]{Posgraduate Project}
%%%%%%%%%%%%%%%%%%%%%%%%%%%%%%%%%%%%%%%%%

\begin{document}
\maketitle

\begin{abstract}
\end{abstract}

\section*{Introduction}
\label{sec:0}


\section{Section 1}
\label{sec:1}

A manifold without boundary $M$ of dimension $ m $  is said to be of class $ C^\infty $ if the differentiability class of the transition maps is $C^\infty.$ In this case, for every $ p\in M $ one can define the tangent space to $ p $, $ T_pM $, as the set of all derivations of the algebra of germs of smooth functions from $ M $ to $ \mathbb{R}  $,  based on $ p $. This definition of tangent space is equivalent (when the differentiability class of $ M $ is $ \infty $) to the equivalence class of germs of curves that passes through $ p $. If a manifold $ M $  is $C^r$, with $r< \infty$, the definition for tangent space is the latter, so we use that definition for the tangent space throughout this text, independently of the differentiability class.

Joining the tangent spaces of every point of the manifold induces the tangent bundle,
$$ TM = \bigsqcup_{p\in M} T_pM, $$ 
equipped with manifold structure given by the one of $ M. $ It is called a bundle because the triple $ (TM, M,\pi:TM \rightarrow {M}) $ is in fact a bundle, where $ \pi:TM \rightarrow {M} $  is the projection function, that assigns each tangent vector to the point where it is based. Each fiber of this bundle is a tangent space based on a point, and the tangent bundle is in particular a vector bundle because each fiber has structure of finite dimensional real vector space. 

For an arbitrary bundle $ (E,B,\pi:E \rightarrow {B}), $ a subbundle of this bundle $ (E',B',\pi':E' \rightarrow {B'}) $  is just a bundle that satisfies $ E'\subset E $, $ B'\subset B $ and $ \pi|_{E'}=\pi'. $ A distribution $ \mathcal{H} $ is a vector subbundle of the tangent bundle, with a fiber inner-product, and the pair of a manifold and a distribution over it defines a subriemannian geometry, as is stated in the following definition.

\begin{df}[Subriemannian Geometry]
	A subriemannian geometry over a manifold $ M $ consists of a distribution $ \mathcal{H}\subset TM $ (referred as horizontal distribution), \textit{i.e.}, a vector subbundle of the tangent bundle of $ M $, together with a fiber inner-product $ <\cdot,\cdot> $ on this subbundle. 
\end{df}

Some examples of this definition are given below.
\begin{ex}[Riemannian Geometry]\label{ex:riem_geo}
	Every riemannian geometry is in particular a subriemannian geometry, where the distribution is all the tangent bundle.
\end{ex}

\begin{ex}[Heisenberg Group]\label{ex:heis_group}
	This is a non-trivial case of subriemannian geometry. Take $ \mathbb{R} ^3 $ as the manifold, and define over it the distribution $ \mathcal{H}\subset T \mathbb{R} ^3 $, whose fiber at an arbitrary $ (x,y,z)\in \mathbb{R}^3  $ is   
	$$ \mathcal{H}_{(x,y,z)} = \left\{ (v_1,v_2,v_3)\in T_{(x,y,z)} \mathbb{R}^3 : v_3- \frac{1}{2} \left( xv_2-yv_1 \right)=0   \right\}. $$ 
	The inner product over $ \mathcal{H}_{(x,y,z)} $ is given by $<\cdot,\cdot>:  \mathcal{H}_{(x,y,z)}\times\mathcal{H}_{(x,y,z)} \rightarrow { \mathbb{R} }:(v,w)\mapsto v_1w_1+v_2w_2,$ where $ v=(v_1,v_2,v_3) $ and $ w=(w_1,w_2,w_3). $  

\end{ex}
	The last example is of particular interest because of its connection with the isoperimetrical problem.
\begin{ex}[Vector Field over a Manifold]\label{ex:vect_field}
	Any vector field $ X: M \rightarrow {TM} $ (seen as a section of the tangent bundle) determines a distribution over the manifold $ M $ whose fiber at an arbitrary $ p\in M $ is the span of $ X_p, $ and the fiber inner-product is the canonical one (given by the isomorphism $ T_pM\cong \mathbb{R} ^m, $ with $\operatorname{dim} \left( M \right) = m$).

	As a particular case of this example, one can consider a system of $ n $ linear differential equations on $ \mathbb{R}^n $, that determines a vector field over $ \mathbb{R}^n,  $ and this in turn generates a distribution via the span of the vectors at each point, with the canonical fiber-inner product.
\end{ex}

Given a subriemannian geometry $ (M, \mathcal{H}), $ a curve or vector field is said to be \textbf{horizontal} if is tangent to $ H $. This means that, in the case of a curve, its tangent vector at every point is contained in the fibers of $ \mathcal{H} $; and in the case of a vector field, its image at every point where it is defined is contained in the fibers of $ \mathcal{H} .$   

For example, every curve over a riemannian geometry is horizontal, as can be deduced from example \ref{ex:riem_geo}. In the example \ref{ex:heis_group}, the liftings of the solutions of the isoperimetrical problem are horizontal. In example \ref{ex:vect_field}, and in particular in the case of systems of linear differential equations, the solution curves of the system of differential equations are examples of horizontal curves on this subriemannian geometry.

In the particular case of the horizontal curves, one can define the \textbf{length of a horizontal smooth curve} $ \gamma $ (denoted by $ \ell(\gamma) $) as in the case of riemannian geometry:
$$ \ell(\gamma) = \int ||\dot{\gamma}|| dt, $$ 
since the tangent vectors of the curve at all points are in the fibers of the horizontal distribution, where there is a inner product defined, and $ ||\dot{\gamma}||= \sqrt[]{<\dot{\gamma}, \dot{\gamma}>}.  $   

With the idea of length of horizontal curves, the \textbf{distance between two points} $ p,q\in M $, denoted by $ d(p,q), $   is defined as the infimum of the lengths of all absolutely continuous\footnote{A curve $ \gamma:I\subset \mathbb{R}  \rightarrow {M} $ is said to be absolutely continuous if it has derivative for almost all $ t\in I. $ } horizontal curves that begin in $ p $ and end in $ q $, that is, 

$$ d(p,q) = \operatorname{inf} \left\{ \ell(\gamma) \ |\ \gamma:[0,1] \rightarrow {M} \text{ is absolutely continuous and }\gamma(0)=p, \ \gamma(1)=q \right\}. $$ 

The distance between two points is said to be infinite if there is no curve joining them. Also, in analogy with riemannian geometry, an absolutely continuous horizontal curve that realizes the distance between two points is called a \textbf{geodesic}. 

Now the natural question that arises is if, given two points on a manifold with a given horizontal distribution, is there a geodesic that joins them? Or more generally, there exists a horizontal curve that joins them? To address this question, the present text will focus on \textbf{smooth distributions}, that is just a horizontal distribution over a manifold such that for each point $ p\in M $, there is an open neighborhood $ U $ of it on which there are smooth vector fields $ X_1,\dots,X_n: U \rightarrow {TM} $ that at each point of $ U $ form a base for the fiber of that point respect to the distribution.

With a smooth distribution $ \mathcal{H} \subset TM, $ it is said that a nonempty immersed submanifold $ N\subseteq M $  is a \textbf{integral manifold} of $\mathcal{H}$ if $ T_p N = \mathcal{H}_p $ for all $ p\in N $ , \textit{i.e.}, if the tangent space of all points of the submanifold is the fiber of the horizontal distribution. As an example, in the subriemannian geometry given in example \ref{ex:vect_field}, the image of any integral curve of the vector field that determines the distribution over the manifold is an integral manifold for that distribution. A particular example of a distribution and its family of integral manifolds is given below.

\begin{ex}[Orthogonal complement of radial vector field]
	Let $ \mathcal{G} $ be the distribution over $ \mathbb{R}^n $ determined by the radial vector field\footnote{This text uses extensively the Einstein's sum convention.} $ x^i \partial / \partial x^i $, and let $ \mathcal{G}^\perp $ be its perpendicular bundle, \textit{i.e.,} the distribution whose fibers are the orthogonal complement of the fibers of $ \mathcal{G}. $ $ \mathcal{G}^\perp $ is a distribution over $ \mathbb{R}^n  $, and the sphere centered on $0$, of radius $ |x| $, for all $ x\neq0 $ is an integral submanifold of $ \mathcal{G}^\perp. $        
\end{ex}

There are cases when a subriemannian geometry over a manifold does not have integral manifolds, as is shown in the following example, taken from \cite{lee2018introduction}.
\bibliographystyle{unsrt}
\nocite{*}
\bibliography{mybibliography}


\end{document}










