%%%% DESCRIPTION %%%%%
% Title of Document: 
% Start Date: 
%%%%%%%%%%%%%%%%%%%%%%%%
%%%FINAL CHECK%%%%%%%%%
%read text
%Bibliography list
%spell check
%chklatex
%unusedlabels
%Overfull

\documentclass[12pt, letterpaper, reqno]{amsart}

\usepackage[utf8]{inputenc}
% indicate the language, for example, spanish or russian
\usepackage[english]{babel}
%\usepackage[spanish]{babel}

%comments
\usepackage{comment}

%drawings
\usepackage{graphicx}

%commutative diagrams
\usepackage[all]{xypic}

\usepackage{import}
\usepackage{pdfpages}
\usepackage{transparent}
\usepackage{xcolor}
\usepackage{mathtools}
\usepackage{breqn}
\usepackage{tikz-cd}
\usepackage{bbold}
%%%%%%%%%%%%%%%%%
\newcommand{\incfig}[2][1]{%
    \def\svgwidth{#1\columnwidth}
    \import{./figures/}{#2.pdf_tex}
}
% why do we need this command?
% \includegraphics does not work? 
% - it does work, but i need to use it
% because the images that i create are in
% .svg format (which provides more advantages),
% so they need to be converted to pdf and inserted
% using that command.
%%%%%%%%%%%%%%%%%%

\pdfsuppresswarningpagegroup=1

%todonotes
\usepackage[colorinlistoftodos,prependcaption,backgroundcolor=white,size=tiny,textwidth=50pt]{todonotes}
\setlength{\marginparwidth}{2cm}

%%%%%%%% DIMENSIONS %%%%%%%%%%%
\textwidth=16truecm
\textheight=22truecm
\topmargin=0pt
\oddsidemargin=0pt
\evensidemargin=0pt
\baselineskip1.2truecm

%%%%%%%% THEOREMS %%%%%%%%%%%%
\theoremstyle{definition}
%\newtheorem{df}{Definition}
\newtheorem{df}{Definition}
\theoremstyle{plain}
\newtheorem{prop}{Proposition}
%\newtheorem{prop}{Proposición}
\newtheorem{thm}{Theorem}
%\newtheorem{thm}{Teorema}
\newtheorem{lm}{Lemma}
%\newtheorem{lm}{Lema}
\newtheorem{cor}{Corollary}
%\newtheorem{cor}{Corolario}
\theoremstyle{remark}
\newtheorem{ex}{Example}
%\newtheorem{ex}{Ejemplo}
\newtheorem{rem}{Remark}
%\newtheorem{rem}{Observación}

%%%%%%%% COMMENTS (REMOVE THEM IN THE FINAL VERSION!!!) %%%%%%%
\def\comment#1{{\par \tt #1}}

%%%%%%%%% FOR DRAFT%%%%%%%%%%%%%
%\renewcommand{\baselinestretch}{1.5}
%%%%%%%%%%%%%%%%%%%%%%%%%%%

%%%%%%%%%%%% PAPER DATA %%%%%%%%%%%%%%%%%%%
\author{Julián Jiménez Cárdenas}
\address{}


\title[Posgraduate Project]{Posgraduate Project}
%%%%%%%%%%%%%%%%%%%%%%%%%%%%%%%%%%%%%%%%%

\begin{document}
\maketitle
\begin{abstract}
\end{abstract}

\section*{Introduction}
\label{sec:0}


\section{Distributions}
\label{sec:1}

\subsection{Distributions and Horizontal Curves}%
\label{sub:distributions}


A manifold without boundary $M$ of dimension $ m $  is said to be of class $ C^\infty $ if the differentiability class of the transition maps is $C^\infty.$ In this case, for every $ p\in M $ one can define the tangent space to $ M $ at $ p $, $ T_p M $, as the set of all derivations of the algebra of germs of smooth functions from $ M $ to $ \mathbb{R}  $,  based at $ p $. This definition of tangent space is equivalent (when the differentiability class of $ M $ is $ \infty $) to the equivalence class of germs of curves that passes through $ p $. If a manifold $ M $  is $C^r$, with $r< \infty$, the definition for tangent space is the latter, so we use that definition for the tangent space throughout this text, independently of the differentiability class.

\begin{df}(\cite{luke2013vector}, section 1.3)\label{def:smooth_vector_bundle}
	The triple $ (E, M, \pi: E \rightarrow {M}) $ is a \textbf{smooth vector bundle} if $ E,M $ are smooth manifolds, $ \pi $ is a smooth surjection; the fibers $ \pi^{-1}(p) $ for $ p\in M $  have the structure of finite dimensional vector spaces; and for every point $ p\in M $, there is an open neighborhood $ U $ of $ p $, a natural number $ k $ and a diffeomorphism $ \phi: U\times \mathbb{R}^k \rightarrow  \pi^{-1}(U),$ such that for all $ q\in U $ the following propositions hold:  

	\begin{itemize}
		\item $ (\pi\circ \phi)(q,v)=q $ for all $ v\in \mathbb{R}^k $, and  
		\item the map $ v \mapsto \phi(p,v) $ is a linear isomorphism between the vector spaces $\pi^{-1}(p) $ and $ \mathbb{R}^k. $     
	\end{itemize}
	The first condition means that the following diagram is commutative,	
	\begin{center}
		\begin{tikzcd}
		U\times \mathbb{R}^k \arrow[d, "p"'] \arrow[rr, "\phi"] &  & \pi^{-1}(U) \arrow[d, "\pi"] \\
		U \arrow[rr, "Id"']                                     &  & U                           
		\end{tikzcd}
	\end{center}
	where $ p:U\times \mathbb{R}^k \rightarrow U, \ p(q,v)=q$ is the projection onto the first factor.
\end{df}
We take the disjoint union of the tangent spaces at every point of the manifold in order to define the tangent bundle,
$ TM = \bigsqcup_{p\in M} T_pM, $ 
equipped with manifold structure given by the one of $ M. $ It is called a bundle because the triple $ (TM, M,\pi:TM \rightarrow {M}) $ satisfies the definition of smooth vector bundle, where $ \pi:TM \rightarrow {M} $  is the projection function, that assigns each tangent vector to the point where it is based. Each fiber of this bundle is a tangent space based on a point, and the tangent bundle is in particular a vector bundle because each fiber has structure of finite dimensional real vector space.



There is another important class of smooth vector bundles, namely, the tensor bundles, that are defined following \cite{wendl2008lecture}, examples 2.4 and 2.5.
\begin{df} 
	If $ (E,M,\pi:E \rightarrow {M}) $ is a smooth vector bundle, the dual bundle of this bundle is another smooth vector bundle $ (E^*,M,\pi^*:E^*\rightarrow {M}) $, known as \textbf{ dual bundle}, whose fibers are the dual spaces to the fibers of $ E, $ that is,
	$$ (\pi^{*})^{-1}(p)=(\pi^{-1}(p))^*, \text{ for all }p\in M,$$ 
	and $$ E^* =\bigsqcup_{p\in M} (\pi^*)^{-1}(p). $$ 
\end{df}
\begin{df}
	If $ (E,M,\pi:E \rightarrow {M}) $ is a smooth vector bundle, the \textbf{$ (k,l)- $tensor bundle} associated to this bundle, with $ k,l\geq0 $, is defined as the smooth vector bundle $ (T^k_l E, M, \pi_{kl}: T^k_l E \rightarrow {M}) $, whose fibers are the vector spaces of $ (k,l) -$tensors of the fibers of $ E, $ that is,     
	$$ \pi_{kl}^{-1}(p) = T^k_l \pi^{-1}(p)=\left( \bigotimes_{j=1}^l (\pi^{*})^{-1}(p)\right)\otimes \left( \bigotimes_{j=1}^k \pi^{-1}(p) \right), \text{ for all }p\in M, $$ 
	and
	$$ T^k_l E = \bigsqcup_{p\in M} \pi_{kl}^{-1}(p). $$ 
\end{df}
\begin{rem}
	If $ k=1,l=0, $ $T^0_1E=E,$ and if $ k=0,l=1, $ $ T^1_0=E^*. $ We associate $ T^0_0=M\times \mathbb{R} $ to the \textbf{trivial line bundle} if both $ k=l=0. $ 
\end{rem}

The formal proof that both definitions are consistent definitions of smooth vector bundles can be found in \cite{luke2013vector}, section 1.3.
\begin{df}
	A \textbf{smooth section} of a smooth vector bundle $ (E,M,\pi:E \rightarrow {M}) $ is a (smooth) application $ s: M \rightarrow {E} $ such that $ \pi\circ s= \mathbb{1}_{M}. $    
\end{df}
For an arbitrary smooth vector bundle bundle $ (E,B,\pi:E \rightarrow {B}), $ a subbundle of this bundle $ (E',B',\pi':E' \rightarrow {B'}) $  is a smooth vector bundle bundle that satisfies $ E'\subset E $, $ B'\subset B $ and $ \pi|_{E'}=\pi'. $
\begin{df}
	
	A \textbf{distribution} $ \mathcal{H} $ over a manifold $ M $  is a vector subbundle of the tangent bundle $ TM $ . If this vector subbundle is in addition smooth, the distribution is said to be smooth, and the rank $ k $  of this distribution is defined as the dimension of its fibers.

	The fiber of $ \mathcal{H} $ at a point $ p\in M $ is denoted as $ H_p. $ 
\end{df}

The concept of (smooth) distribution can be defined in an alternative but equivalent (see \cite{lee2003introduction}, Ch. 19, lemma 19.5) fashion using 1-forms, with the help of the following proposition.

\begin{prop}[Smooth Distribution (using 1-forms)]
	Suppose that $ M $ is a smooth $ m- $dimensional manifold, and $ \mathcal{H} \subset TM$ is a distribution of rank $k$. Then $ \mathcal{H} $ is smooth if and only if for every point $ p\in M $ there is a neighborhood $ U $ of that point where exist $n-k$ smooth 1-forms $ \omega^1,\dots,\omega^{n-k}, $ such that for all $ q\in U $,

	$$ \mathcal{H}_q = \operatorname{Ker} \omega^1|_q\cap \cdots \cap\operatorname{Ker} \omega^{n-k}|_q.  $$ 
\end{prop}

The following proposition gives an alternative way to represent a smooth distribution in terms of local frames (\cite{lee2003introduction}, Ch. 19, page 491).

\begin{prop} [Smooth Distribution (using local frames)]
	It is said that $ \mathcal{H}\subset TM $ is a smooth distribution over $M  $ if and only if  for each point $ p\in M $, there is an open neighborhood $ U $ of it on which there are smooth vector fields $ X_1,\dots,X_n: U \rightarrow {TM} $ that at each point of $ U $ form a base for the fiber of that point respect to the distribution.
\end{prop}
Some examples of distributions over manifolds are given below.
\begin{ex}[Tangent Bundle]
	For any manifold $ M $, the tangent bundle $ TM $ is by definition a distribution over it. 
\end{ex}
\begin{ex}[Heisenberg Group]\label{ex:heis_group}
	Take $ \mathbb{R} ^3 $ as the manifold, and define over it the distribution $ \mathcal{H}\subset T \mathbb{R} ^3 $, whose fiber at an arbitrary $ (x,y,z)\in \mathbb{R}^3  $ is   
	$$ \mathcal{H}_{(x,y,z)} = \operatorname{Ker} \omega|_{(x,y,z)},$$
	where $ \omega = dz-(x dy-y dx)/2\in\Omega^1( \mathbb{R}^3).$ It is important to notice that $ \omega\neq 0 $ for all $ (x,y,z)\in \mathbb{R}^3. $ 
\end{ex}
\begin{rem}
	The last example is of particular interest because of its connection with the isoperimetrical problem. For more information see \cite{montgomery2002tour}, Ch. 1.
\end{rem}
\begin{ex}[Vector field over a manifold]\label{ex:vect_field}
Any smooth vector field $ X: M \rightarrow {TM} $ (which does not vanish out at any point) on a manifold determines a distribution $ \mathcal{H}$ over the manifold $ M $ whose fiber at an arbitrary $ p\in M $ is the span of $ X_p. $
\end{ex}


Given a distribution $ \mathcal{H} $  over a manifold $M$,  a curve or vector field is said to be \textbf{horizontal} if it is tangent to $ \mathcal{H} $. This means that, in the case of a curve, its tangent vector at every point is contained in the fibers of $ \mathcal{H} $; and in the case of a vector field, its value at every point is contained in the fibers of $ \mathcal{H} .$   

For example, every curve over a manifold with distribution given by its tangent bundle is horizontal. In the example \ref{ex:heis_group}, the liftings of the solutions of the isoperimetrical problem are horizontal. In example \ref{ex:vect_field}, and in particular in the case of systems of linear differential equations, the solution curves of the system of differential equations are examples of horizontal curves for that distribution.

\subsection{Frobenius' Theorem}%
\label{sub:frobenius_theorem}


Now the natural question that arises is if, given two points on a manifold with a given horizontal distribution, is there a smooth horizontal curve that joins them? 

Given a smooth distribution $ \mathcal{H} \subset TM$, we say that a nonempty immersed submanifold $ N\subseteq M $  is an \textbf{integral manifold} of $\mathcal{H}$ if $ T_p N = \mathcal{H}_p $ for all $ p\in N $ , \textit{i.e.}, if the tangent space of all points of the submanifold is the fiber of the horizontal distribution. As an example, in the distribution given in example \ref{ex:vect_field}, the image of any integral curve of the vector field that determines the distribution over the manifold is an integral manifold for that distribution. A particular example (taken from \cite{lee2003introduction}, example 19.1) of a distribution and its family of integral manifolds is given below.

\begin{ex}[Orthogonal complement of a given vector field]
Let $ \mathcal{H} $ be the distribution over $ \mathbb{R}^n $ determined by the radial vector field $ x^i \partial / \partial x^i $, and let $ \mathcal{H}^\perp $ be its perpendicular bundle, \textit{i.e.,} the distribution whose fibers are the orthogonal complement of the fibers of $ \mathcal{H}. $ $ \mathcal{H}^\perp $ is a distribution over $ \mathbb{R}^n  $, and the sphere centered at $0$, of radius $ |x| $, for all $ x\neq0 $ is an integral submanifold of $ \mathcal{H}^\perp. $        
\end{ex}

There are distributions over a manifold that do not have integral manifolds, as is shown in the following example (taken from \cite{lee2003introduction}, Ch. 19, example 19.1).

\begin{ex}[Distribution without integral manifolds] \label{ex:no_integral_manifolds}
	Consider the distribution $ \mathcal{H} $ over $ \mathbb{R}^3 $ spanned by the following vector fields: 
	$$ X = \frac{\partial}{\partial x} + y \frac{\partial}{\partial z}, \quad Y = \frac{\partial}{\partial y}.  $$ 
	This distribution does not have integral manifolds. To see this, suppose that $ N $ is an integral manifold that contains an arbitrary point $ \textbf{x} \in \mathbb{R}^3.  $ Because $ X $ and $ Y $ are tangent to $ N $, any integral curve of $ X $ or $ Y $ that starts in $ N $ stays in $ N $, at least for a short time (\textit{i.e.}, the integral curve $ \gamma $ must be defined over a sufficiently small interval such that its image is fully contained in $ N $).   

	Therefore, as the integral curve of $ X $ is a straight line contained in the plane that is parallel to the $ xz- $plane, there is a segment of this line that is contained in $ N $, and for every point in this segment, the integral curves of $ Y $ that passes through them are straight lines parallel to the $ y- $axis, so there must be an open neighborhood of $ \textbf{x}  $ that contains a plane generated by the segments of the integral curves of $ X $ and $ Y $ that passes through $ \textbf{x}$ (see figure \ref{fig:no_integral_manifolds}). However, the tangent plane at any point $ p $  of this plane  off of the segment induced by the integral curve of $ X $ is not equal to $ \mathcal{H}_p, $ since $ X $ depends of $ y $.   
\end{ex}

\begin{figure}
    \centering
    \incfig{no_integral_manifolds}
    \caption{Illustration of the planes generated at each point by the integral curves of the smooth vector fields that determine the distribution given in example \ref{ex:no_integral_manifolds}.}
    \label{fig:no_integral_manifolds}
\end{figure}

A smooth distribution $ \mathcal{H} $ is said to be \textbf{involutive} if given any pair of smooth vector fields $ X,Y $ defined on a open subset of $ M $  that satisfy $X_p,Y_p\in \mathcal{H}_p $, their Lie bracket also satisfies the same condition, \textit{i.e.}, it is tangent to the distribution in the given open subset. 

Let $ \Gamma(\mathcal{H}) $ be the set of all smooth vector fields globally defined, tangent to $ \mathcal{H}. $ It is clear from the definition of involutivity that if $ \mathcal{H} $ is involutive, $ \Gamma (\mathcal{H})\subset \mathfrak{X}(M) $ is a Lie subalgebra. The converse statement (if $ \Gamma(\mathcal{H}) $ is a Lie algebra, then $ \mathcal{H} $ is involutive) is also true, and can be proved extending any pair of smooth vector fields locally defined, tangent to $ \mathcal{H} ,$ to a pair of vector fields in $ \Gamma( \mathcal{H}), $ using an adequate bump function. 

A smooth distribution $ \mathcal{H} $ over a manifold $ M $ is called \textbf{integrable} if each point of $ M $ is contained in an integral manifold of $ \mathcal{H}. $  It is clear then that \textit{every integrable distribution is involutive}, because every pair of smooth vectors tangent to $ \mathcal{H}, $ defined over an open set $ U\subset M $, satisfy that their Lie bracket is also tangent to $ \mathcal{H} $ in that open set, since there exists a integral manifold $ N $  of $ \mathcal{H} $ for every point of $ U $ such that the pair of smooth vectors are tangent to $ N $, and in consequence, their Lie bracket is also tangent to $ N. $  

In fact, the involutivity condition does not need to be checked for all smooth vector fields on $ \Gamma( \mathcal{H}), $ but it suffices to check if the Lie brackets of pairs of smooth vector fields of the local frame that determines the distribution are tangent to it, as the following lemma states (\cite{lee2003introduction}, lemma 19.4).

\begin{lm}[Local frame criterion for involutivity] 
	Let $ \mathcal{H}\subset TM $ be a distribution. If for every point in $ M $ there is a neighborhood such that there exists a local smooth frame $ (X_1,\dots, X_k) $ for $ \mathcal{H} $ such that $ [X_i,X_j] $ is a section of $ \mathcal{H} $ for each $ i,j, $ then $ \mathcal{H} $ is involutive.
\end{lm}

\begin{df}
	Given a smooth distribution $ \mathcal{H}\subset TM $  of rank $ k, $ it is said that a smooth coordinate chart $ (U,\phi) $ of $ M $ is \textbf{flat for $ \mathcal{H} $ } if $ \phi(U) $ is a cube in $ \mathbb{R}^m  $ (being $ m $ the dimension of $ M $ ), and at points of $ U $, $ \mathcal{H} $ is spanned by the first $ k $ coordinate vector fields $ \partial/\partial x^1,\dots,\partial/\partial x^k $ (see figure \ref{fig:fittogether}).
\end{df}
The geometrical meaning of this property for a distribution is that, locally (in that chart), the manifold is homeomorphic to a cube in $ \mathbb{R}^m,  $ and the vector subspaces induced by the distribution at each point fit together as affine spaces of dimension given by the rank of the distribution in $ \mathbb{R}^m $. Also, each slice of the form $ x^{k+1}=c^{k+1},\dots, x^m=c^m,$ for constants $ c^{k+1}, \dots, c^m $ in the image of $ U $ of such a chart is an integral manifold of the distribution generated by $ \partial/\partial x^1,\dots,\partial/\partial x^k $, and the preimage of this integral manifold is an integral manifold of $ \mathcal{H}, $ contained in $ U. $  

\begin{figure}
    \centering
    \incfig{fittogether}
    \caption{Flat chart for a distribution.}
    \label{fig:fittogether}
\end{figure}

The case when every point in $ M $ has a coordinate chart centered in it, flat for $ \mathcal{H} $ is discussed in the following definition and geometrical interpretation.

\begin{df}
	A smooth distribution $ \mathcal{H}\subset TM $  is said to be \textbf{completely integrable} if there exists a flat chart for $ \mathcal{H} $ in a neighborhood of each point of $ M. $ 
\end{df}

Therefore, if a distribution is completely integrable, is then integrable, since there exists an integral manifold for each point (that in particular is known as \textbf{maximal integral manifold}, as the dimension of this manifold coincides with the rank of the distribution), given by the integral manifold that contains this point in the image of the coordinate chart flat for $ \mathcal{H}. $ In this way, \textit{ if a distribution is completely integrable, it is integrable, and therefore involutive.} In fact, these implications are actually equivalences, thanks to Frobenius's theorem.

\begin{thm}[Frobenius]
	Every involutive distribution is completely integrable.	
\end{thm}

\begin{proof}
	(\cite{lee2003introduction}, Ch. 19, theorem 19.12) First, it will be shown that any involutive distribution is spanned by independent smooth commuting vector fields, and as a consequence of this fact, the distribution is completely integrable.

	Let $ \mathcal{H} $ be an involutive distribution of rank $ k $ on an $ m- $dimensional manifold $ M $, and let $ p\in M $ . Let $ (U,\varphi) $ be a smooth coordinate chart centered in $ p. $ Then, $ \varphi(U)\subset \mathbb{R}^m $, and let $ X_1,\dots,X_k $ be a smooth local frame for $ \mathcal{H} $ in $ U. $

	Now, $ \left\{ d\varphi \left( X_i \right) \right\}_{i=1}^k $ is a set of linearly independent vector fields over $ \varphi(U). $ One can reordinate the coordinates of $ \mathbb{R}^m $ to make $ \mathcal{H}'_{\varphi(p)} $ (understood as the fiber of the distribution whose local frame is given by $ \left\{ d\varphi \left( X_i \right) \right\}_{i=1}^k $) complementary to the subspace generated by $ \left( \partial/\partial x^{k+1}|_{\varphi(p)},\dots,\partial/\partial x^{m}|_{\varphi(p)} \right). $   

	Let $ \pi: \mathbb{R}^m \rightarrow  \mathbb{R}^k$ the projection onto the first $ k $ coordinates, that is, $ \pi(x^1,\dots,x^m)=(x^1,\dots,x^k). $ This map induces a smooth (smooth because it is the composition $\mathcal{H}' \hookrightarrow T\varphi(U) \xrightarrow[]{d\pi} T\pi(\varphi(U))$ )  bundle homomorphism $ d\pi: T \mathbb{R}^m \rightarrow \mathbb{R}^k	 $ that acts as follows:
	$$ d\pi \left( \sum_{i=1}^m v_i \frac{\partial}{\partial x^i} \Big|_{\varphi(q)}   \right) = \sum_{i=1}^k v_i \frac{\partial}{\partial x^i} \Big|_{\pi(\varphi(q))}, \text{ for }q\in U.$$ 

	By the choice of coordinates, $ \mathcal{H}'_{\varphi(p)}\subset T_{\varphi(p)} \mathbb{R}^m $ is complementary to the kernel of $ d\pi_{\varphi(p)}, $ so the restriction $ d\pi|_{\mathcal{H}'_{\varphi(p)}} $ is bijective. By continuity, the same is true for $ d\pi|_{\mathcal{H}'_{\varphi(q)}} $, with $ q\in U $, and therefore, the matrix entries of $ \left(d\pi|_{\mathcal{H}'_{\varphi(q)}} \right)^{-1}: T_{\pi(\varphi(q))} \mathbb{R}^k \rightarrow \mathcal{H}'_{\varphi(q)}$ are smooth, which makes the map smooth. With this in mind, the set of vector fields $ V_1,\dots,V_k $, defined by
	$$ V_i|_{\varphi(q)}= \left(  d\pi|_{\mathcal{H}'_{\varphi(q)}}  \right)^{-1} \frac{\partial}{\partial x^i} \Big|_{\pi(\varphi(q))} $$ 
	are a local frame of $ \mathcal{H'}. $ It is in fact a smooth commuting local frame, as will be seen below. 

	First, notice that $ V_i $ and $ \partial/\partial x^i $ are $ \pi- $related, for $ i=1,\dots,k, $ because
	$$ \frac{\partial}{\partial x^i} \Big|_{\pi(\varphi(q))} = d\pi |_{\mathcal{H}'_{\varphi(q)}} \left( V_i |_{\varphi(q)} \right) = d\pi_{\varphi(q)}(V_i|_{\varphi(q)}), $$ 
	so by the naturality of Lie brackets,
	$$ d\pi_{\varphi(q)} \left( \left[ V_i,V_j \right]_{\varphi(q)} \right)= \left[ \frac{\partial}{\partial x^i}, \frac{\partial}{\partial x^j}   \right]_{\pi(\varphi(q))}=0, $$ 
	but as $ d\pi|_{\mathcal{H}'_{\varphi(q)}} $ is injective and $ [V_i,V_j]_{\varphi(q)}\in \mathcal{H}'_{\varphi(q)} $, since $ \mathcal{H} $ is involutive, $ [V_i,V_j]_{\varphi(q)}=0 $ for all $ q\in U, $ so $ \left\{ V_i \right\}_{i=1}^k $ form a smooth commuting frame for $ \mathcal{H}', $ and $ \left\{ d\varphi^{-1}(V_i) \right\}_{i=1}^k $ are a smooth commuting local frame for $ \mathcal{H}. $   

	Finally, to show that the existence of the commuting frame $ \left\{V_i \right\}_{i=1}^k $ for $ \mathcal{H}' $ is a sufficient condition for $ \mathcal{H} $ to be completely integrable, let $ \theta_i $ be the flow of $ V_i, $ $i=1,\dots,k.$ There is a neighborhood $ W $ of $ \varphi(p)  $, contained in $ \varphi(U) $ such that the composition  
	$ (\theta_1)_{t_1}\circ(\theta_2)_{t_2}\circ\cdots\circ(\theta_k)_{t_k} $
	is well defined, for sufficiently small $ t_1,\dots,t_k $ ($|t_i|<\epsilon  $, for an adequate $ \epsilon>0$). Define $ \Omega \subset \mathbb{R}^{m-k} $ as    
	$$ \Omega = \left\{ (s^{k+1},\dots,s^m)\in \mathbb{R}^{m-k}\ : \ (0,\dots,0,s^{k+1},\dots,s^m)\in W \right\}, $$ 
	and $\Phi:(-\epsilon,\epsilon)\times \Omega \rightarrow \varphi(U)$ as
	$$ \Phi(s^1,\dots,s^k,s^{k+1},\dots,s^m)=(\theta_1)_{s^1}\circ\cdots\circ(\theta_k)_{s^k}(0,\dots,0,s^{k+1},\dots,s^m). $$ 
	
	Notice that, by construction, $ \Phi( \left\{ 0 \right\}^k\times\Omega) = ({0}^k\times \mathbb{R}^{m-k})\cap W. $ Moreover, $ \partial/\partial s^i $ and $ V_i $ are $ \Phi- $related for $ i=1,\dots,k, $ because, as the flows commute, for a given $ s^0=(s^1,\dots,s^m)\in (-\epsilon,\epsilon)^k\times \Omega, $ 

	\begin{dmath*}
		d\Phi_{s^0} \left( \frac{\partial}{\partial s^i} \Big|_{s^0}  \right) f = \frac{\partial}{\partial s^i} \Big|_{s^0} f \left( \Phi \left( s^1,\dots,s^m \right) \right) = \frac{\partial}{\partial s^i} \Big|_{s^0} f \left( (\theta_1)_{s^1}\circ\cdots\circ(\theta_k)_{s^k}(0,\dots,0,s^{k+1},\dots,s^m) \right) = \frac{\partial}{\partial s^i} \Big|_{s^0} f \left( (\theta_i)_{s^i}\circ\cdots\circ(\theta_{i-1})_{s^{i-1}}\circ(\theta_{i+1})_{s^{i+1}}\circ\cdots\circ(\theta_k)_{s^k}(0,\dots,0,s^{k+1},\dots,s^m) \right),
	\end{dmath*}
	and for any $ q\in\varphi(U) $, $ t\mapsto (\theta_i)_t(q) $ is an integral curve of $ V_i, $ so the above expression is equal to $ V_i|_{\Phi(s^0)} f $, which shows that $ \partial/\partial s^i |_{s^0} $ and $ V_i $ are $ \Phi- $related.   

	By the previous computations, $$ d\Phi_0\left( \frac{\partial}{\partial s^i}\Big|_{0} \right) = V_i |_{\varphi(p)}, \ i=1,\dots,k, $$ and on the other hand, since $ \Phi(0,\dots,0,s^{k+1},\dots,s^m)=(0,\dots,0,s^{k+1},\dots,s^m), $ it follows that
	$$ d\Phi_0 \left( \frac{\partial}{\partial s^i}\Big|_{0}  \right) = \frac{\partial}{\partial x^i}\Big|_{\varphi(p)},\ i=k+1,\dots,m.  $$ 
	
	Therefore, $ d\Phi_0 $ takes the basis $ \left( \partial/\partial s^1 |_0,\dots,\partial/\partial s^m |_0 \right) $ of $ T_0 \mathbb{R}^m $ to the basis $ ( V_1|_{\varphi(p)},  \dots,$ $ V_k|_{\varphi(p)}, \partial/\partial x^{k+1}|_{\varphi(p)}, \dots, \partial/\partial x^{m}|_{\varphi(p)}) $ of $ T_{\varphi(p)} \varphi(U). $ By the inverse function theorem, $ \Phi $ is a local diffeomorphism (in a neighborhood of $ 0 $), and $ \phi=\Phi^{-1} $ is a smooth coordinate chart that takes $ V_i $ to $ \partial/\partial s^i $, for $ i=1,\dots,k, $ and takes $ \partial / \partial x^{i} $ to $ \partial/\partial s^i $, for $ i=k+1,\dots,m. $ Thus, the smooth coordinate chart flat for $ \mathcal{H} $ in a neighborhood of $ p\in M $ is $ \phi\circ\varphi, $ and since this is independent of $ p $, for every point there is a smooth coordinate chart flat for $ \mathcal{H}, $ what makes $ \mathcal{H} $ a completely integrable distribution.

\end{proof}

The next proposition is one of the main consequences of Frobenius theorem, and it is fundamental to study foliations, topic that is going to be discussed below.
\begin{prop}[Local structure of integral manifolds]\label{prop:local_structure}
	Let $ \mathcal{H} $ be an involutive distribution of rank $ k $ on a smooth manifold $ M $, and let $ (U,\varphi) $ be a flat chart for $ \mathcal{H} $. If $ N $ is any connected integral manifold of $ \mathcal{H}, $ then $ \varphi(U\cap N) $ is the union of countably many disjoint open subsets of parallel $ k- $dimensional slices of $ \varphi(U) $, whose preimages are open in $ N $ and embedded in $ M $.
\end{prop}

\begin{figure}
    \centering
    \incfig{local_structure}
    \caption{Local structure of an integral manifold.}
    \label{fig:local_structure}
\end{figure}

\begin{proof}(\cite{lee2003introduction}, Ch. 19, proposition 19.16)
	Let $ N $ be an integral manifold of $ \mathcal{H}. $ Since the inclusion map $ \iota: N \hookrightarrow M $ is continuous, $ \varphi( \iota^{-1}(U))=\varphi(N\cap U) $  is open in $ \varphi(N) $. Then, $ \varphi(U\cap N) $ is the union of open slices (\textit{i.e.}, open subsets of a slice of $ \varphi(U)$) with $ x_i=\text{constant} $, for $ i=k+1,\cdots,m.$ Moreover, this union is at most countable, because $ N $ is second countable, and since $ \varphi(N\cap U) $ is a union of open slices, $ \pi(\varphi(U\cap N)) $ consists of a countable number of points (recall that $ \pi: \varphi(U) \rightarrow \mathbb{R}^{m-k} $ is the projection onto the last $ m-k $ coordinates) in $ \mathbb{R}^{m-k}, $ which implies that the union of open slices is at most countable.  

	Finally, if $ V $ is an open slice in the slice $ S, $ the inclusion map $ \varphi^{-1}(V) \hookrightarrow M $ is a smooth embedding, because it is the composition of smooth embeddings $ \varphi^{-1}(V)\hookrightarrow \varphi^{-1}(S) \hookrightarrow M.$ 
\end{proof}

The last proposition implies that one can put all the maximal integral manifolds of an involutive distribution of rank $ k $  together, to obtain a partition on $ M $ into $ k- $dimensional submanifolds, that satisfy the following definition.
\begin{df}
	A smooth chart $ (U,\varphi) $ for $ M $ is called \textbf{flat for a collection $ \mathcal{F} $ of $ k- $dimensional submanifolds of $ M $} if $ \varphi(U) $ is a cube in $ \mathbb{R}^m, $ and the image of each submanifold via $ \varphi $ intersects $ \varphi(U) $ in either the empty set or in a countable union of $ k- $dimensional slices of the form $ x^{k+1}=c^{k+1},\dots,x^m=c^m. $   
\end{df}

\begin{df}
	A \textbf{foliation of dimension $ k $ on a smooth manifold $ M $} is a collection $ \mathcal{F} $ of disjoint, connected, nonempty, immersed $ k- $dimensional submanifolds of $ M $ (called the \textbf{leaves} of the foliation), whose union is $ M $, and such that in a neighborhood of each point $ p\in M $ there is a flat chart for $ \mathcal{F}. $  
\end{df}

The following are examples of foliations over a manifold, mainly taken from \cite{lee2003introduction}, Ch. 19, example 19.18.

\begin{ex}[Collection of affine subspaces]
	The collection of all $ k- $dimensional affine subspaces of $ \mathbb{R}^m $ parallel to $ \mathbb{R}^k\times \left\{ 0 \right\} $ is a $ k- $dimensional foliation for $ \mathbb{R}^m. $ 
\end{ex}

\begin{ex}[Spheres centered at the origin]
	The collection of all spheres centered at $ 0 $ is an $ (m-1)- $dimensional foliation of $ \mathbb{R}^m\setminus \left\{ 0 \right\} $. 
\end{ex}

\begin{ex}[Cartesian product of manifolds]\label{ex:product_foliation}
	If $ M $ and $ N $ are connected smooth manifolds, the collection of subsets of the form $ M\times \left\{ q \right\}, $ with $ q\in N $, is a foliation of $ M\times N, $ each of whose leaves is diffeomorphic to $ M. $  	
\end{ex}

\begin{ex}[Foliations on a torus]
	The torus $ T= \mathbb{S}^1\times \mathbb{S}^1 $ can be endowed with the distribution induced by the cartesian product of manifolds (see example \ref{ex:product_foliation}). In this case, the foliation is conformed by copies of $ \mathbb{S}^1. $ The horizontal curves are segments of this copies, and if two points lay in different copies, there is not a horizontal curve that connects them.
\end{ex}

From the definition of foliations, it is clear that \textit{if $ \mathcal{F} $ is a foliation on a smooth manifold $ M, $ the collection of tangent spaces to the leaves of $ \mathcal{F} $ forms an involutive distribution on $ M. $} In a reciprocal way, the collection of maximal integral manifolds of an involutive distribution forms a foliation over the manifold, as it is stated in the global Frobenius theorem.

\begin{thm}[Global Frobenius theorem]
	Let $ \mathcal{H} $ be an involutive distribution on a smooth manifold $ M. $ The collection of all maximal connected integral manifolds of $ \mathcal{H} $ forms a foliation of $ M. $ 
\end{thm}

The next lemma (\cite{lee2003introduction}, Ch. 19, lemma 19.22) is going to be used to prove this theorem.

\begin{lm}\label{lm:man_struct}
	Suppose that $ \mathcal{H}\subset TM $ is an involutive distribution, and let $ \left\{ N_\alpha \right\}_{\alpha\in A} $ be any collection of connected integral manifolds of $ \mathcal{H} $ with a point in common. Then, $ N=\bigcup_{\alpha} N_\alpha $ has a unique smooth manifold structure making it into a connected integral manifold of $ \mathcal{H}. $  
\end{lm}

\begin{proof}[Proof of the global Frobenius theorem]
	(\cite{lee2003introduction}, Ch. 19, theorem 19.21) For each $ p\in M $, let $ L_p $ be the union of all connected integral manifolds of $ \mathcal{H} $ containing $ p. $ By lemma \ref{lm:man_struct}, $ L_p $ is a connected integral manifold of $ \mathcal{H} $ containing $ p $, and it is clearly maximal. By maximality, if $ L_p $ and $ L_{p'} $ intersect, $ L_p=L_{p'} $, because $ L_p\cup L_{p'} $ is an integral manifold containing both $ p $ and $ p' $. Thus, the maximal connected integral manifolds are either disjoint or identical.    

	If $ (U,\varphi) $ is any flat chart for $ \mathcal{H}, $ then $ \varphi(L_p\cap U) $ is a countable union of open subsets of slices (by proposition \ref{prop:local_structure}). For any such slice $ S, $ if $ \varphi(U\cap L_p)\cap S $ is neither empty nor all of $ S, $ then $ L_p\cup \varphi^{-1}(S )$ is a connected integral manifold properly containing $ L_p, $ which contradicts the maximality of $ L_p. $ Therefore, $ \varphi(L_p\cap U) $ is precisely a countable union of slices, so the collection $ \left\{ L_p\ : \ p\in M \right\} $ is the desired foliation.
\end{proof}

With the global Frobenius theorem, one can answer the question at the start of this subsection: in general, for an given distribution there is no smooth horizontal curve that connects an arbitrary pair of points, because the points can be in different leaves of the foliation given by the distribution.


\section{Subriemannian Geometry}%
\label{sec:subriemannian_geometry}

\subsection{Subriemannian Structure and Geodesics}%
\label{sub:subriemannian_structure_and_geodesics}
\begin{df}
	A \textbf{subriemannian structure over a manifold} $ M $ is a pair $ (\mathcal{H}, <\cdot,\cdot>)$, where $ \mathcal{H}\subset TM $ is a distribution and $ <\cdot,\cdot> $ is a section of the bundle $ T^0_2 \mathcal{H} \xrightarrow[]{\pi} M, $ whose values are positive definite symmetric bilinear forms (see \cite{hatcher2003vector}, Ch. 1, page 12).

	The distribution $ \mathcal{H} $ is called \textbf{horizontal}, and the tensor field $ <\cdot,\cdot> $ is called \textbf{metric}.  
\end{df}

Some examples of subriemannian structures are given below.
\begin{ex}[Riemannian Structure]\label{ex:riem_geo2}
	Every riemannian structure is in particular a subriemannian structure, where the distribution is the entire tangent bundle.
\end{ex}

\begin{ex}[Heisenberg Group]\label{ex:heis_group2}
	The distribution of the Heisenberg group is commented in example \ref{ex:heis_group}. The inner product over a fiber $ \mathcal{H}_{(x,y,z)} $, with $ (x,y,z)\in \mathbb{R}^3 $  is given by $<\cdot,\cdot>:  \mathcal{H}_{(x,y,z)}\times\mathcal{H}_{(x,y,z)} \rightarrow { \mathbb{R} }:(v,w)\mapsto v_1w_1+v_2w_2,$ where $ v=(v_1,v_2,v_3) $ and $ w=(w_1,w_2,w_3). $  

\end{ex}
\begin{ex}[Vector Field over a Manifold]\label{ex:vect_field2}
	As seen in example \ref{ex:vect_field}, any smooth vector field $ X: M \rightarrow TM$ that does not cancel out in any point determines a distribution.  The fiber inner-product $ <\cdot,\cdot>: \mathcal{H}_p \times \mathcal{H}_p \rightarrow \mathbb{R} $ for $ p\in M $  is given by $<\lambda_1 X_p,\lambda_2 X_p>=\lambda_1 \lambda_2.$
\end{ex}
In the particular case of the horizontal curves, one can define the \textbf{length of a horizontal smooth curve} $ \gamma $ (denoted by $ \ell(\gamma) $) as in the case of riemannian geometry:
$$ \ell(\gamma) = \int ||\dot{\gamma}|| dt, $$ 
since the tangent vectors of the curve at all points are in the fibers of the horizontal distribution, where there is a inner product defined, and $ ||\dot{\gamma}||= \sqrt[]{<\dot{\gamma}, \dot{\gamma}>}.  $   

\begin{df}
	The \textbf{distance between two points} $ p,q\in M $, denoted by $ d(p,q), $   is defined as the infimum of the lengths of all absolutely continuous horizontal curves that begin in $ p $ and end in $ q $, that is, 
$$ d(p,q) = \operatorname{inf} \left\{ \ell(\gamma) \ |\ \gamma:[0,1] \rightarrow {M} \text{ is absolutely continuous and }\gamma(0)=p, \ \gamma(1)=q \right\}. $$ 
The distance between two points is said to be infinite if there is no horizontal curve joining them.
\end{df}

The curves that are considered in the previous definition of distance are required to be absolutely continuous, that is a weaker condition of smoothness, and refers to a curve that is differentiable in almost all points of its domain. Nevertheless, the distance remains invariant if we consider instead the set of all smooth curves (see \cite{montgomery2002tour}, page 23), but the motivation to consider the bigger set of absolutely continuous curves is the fact that the curves that realize the distance between two points are not always smooth.

In analogy with riemannian geometry, an absolutely continuous horizontal curve that realizes the distance between two points is called a \textbf{geodesic}. 


\subsection{Chow's Theorem}%
\label{sub:chow_s_theorem}

The section \ref{sub:frobenius_theorem} provides a class of (involutive) distributions where it is not true that every two points can be connected through a horizontal curve.  Now, it is worthwhile to ask for a sufficient condition for a distribution to have the connectedness property through horizontal curves. To discuss such a condition, we need to cover some preliminaries. 

\begin{df}
	A distribution $ \mathcal{H}\subset TM $ is called \textbf{bracket generating} if for every $ p\in M $, there is a local frame $ X_1,\dots,X_k: U \rightarrow {TM} $ of $ \mathcal{H} $ such that 
	$$ TU = \operatorname{span}\left\{ [X_{i_1},\dots,[X_{i_{j-1}},X_{i_j}]]\ : \ i_1,\dots,i_j=1,\dots,k;\ j\in \mathbb{N} \right\}.  $$ 
\end{df}

\begin{lm}\label{lm:approx}
	Let $ X_1,X_2 $ be two smooth vector fields over the manifold $ M $, with respective local flows $ \Phi_1,\Phi_2:(-\epsilon,\epsilon)\times U \rightarrow {U}$, and let $ p\in U. $ Then, in any coordinate system the following relation holds:
	$$ \left[ \Phi_1(t), \Phi_2(t) \right](p) := \left( \Phi_1(t)\circ \Phi_2(t) \circ \Phi_1(t)^{-1}\circ \Phi_2(t)^{-1} \right)(p)=p+t^2 \left[ X_1,X_2 \right](p) + O(t^3). $$ 
\end{lm}
\begin{proof}
	By Taylor's theorem, we have $$ \Phi_i(t)(p)=p+tX_i(p)+ \frac{t^2}{2} X_i(X_i(p)) + O(t^3), \text{ for }i=1,2.  $$ 
	In the other hand, as $ \left\{ \Phi_i(t) \right\}_{t\in(-\epsilon,\epsilon)} $ is the family of diffeomorphisms associated with the flow of $ X_i, $ $ \Phi_i(t)^{-1}=\Phi_i(-t). $ Then,   
	\begin{dmath*}
	 (\Phi_1(t)^{-1}\circ \Phi_2(t)^{-1})(p) = p- t \left( X_1(p)+X_2(p) \right)+ t^2 X_1(X_2(p)) + t^2 \left( X_1(X_1(p))+ X_2(X_2(p)) \right) + O(t^3), 
	\end{dmath*}
	and evaluating $ \Phi_1(t)\circ\Phi_2(t) $ in this point gives

	\begin{dmath*}
	\left( \Phi_1(t)\circ \Phi_2(t) \circ \Phi_1(t)^{-1}\circ \Phi_2(t)^{-1} \right)(p) = p+t^2 \left[ X_1,X_2 \right](p)+ t^2 \left( X_1(X_1(p)) + X_2(X_2(p)) \right) - t( X_1(p)+X_2(p)) - t^2 \left( X_1(X_1(p)) + X_2(X_2(p)) \right) +t( X_1(p)+X_2(p)) + O(t^3) = p+t^2 \left[ X_1,X_2 \right](p) + O(t^3), 
	\end{dmath*}
	which is the desired result.	
\end{proof}

Chow's theorem guarantees that any bracket generating over a connected manifold has the connectedness property through horizontal curves, and we going to derive this result in the rest of this section, following closely the procedure of \cite{montgomery2002tour}, Section 2.4. First of all, choose a base point $ p\in M $ and a local orthonormal frame $ X_i $, $ i=1,\dots, k $ for the distribution $ \mathcal{H} $. Let $ \Phi_i $ be their respective flows. These flows can be used to move in the horizontal directions (points connected to $ p $ through integral curves of the orthonormal frame), using the formula $ \Phi_i(t)(q)=q+tX_i(q)+O(t^2) $ for sufficiently small $ t. $ 

The assignments $ t\mapsto \Phi_i(t)(q) $ are horizontal curves for the distribution $ \mathcal{H}, $ and are called \textbf{simple horizontal curves.} As the frame is orthonormal, the length of a simple horizontal curve with $ 0\leq t\leq \epsilon  $ is $ \epsilon. $   

Applying $ \Phi_k(t_k)\circ\cdots\circ \Phi_2(t_2)\circ\Phi_1(t_1)$ to $ p $ and letting $ t_1,\dots,t_k $ vary over the cube $ |t_i|\leq\epsilon, $ we move from $ p $ to another point inside the $ k- $dimensional  cube of volume $ (2\epsilon)^k$ in the coordinate chart, whose sides are determined by the local frame in the image of $ p $. Each point of this cube can be reached from $ p $ through the concatenation of $ k $ or fewer flows, and this cube is inside of the subriemannian ball of radius $ k\epsilon $, that is,
\begin{align*}
 \operatorname{Box}_p(\epsilon):= \left\{ \Phi_k(t_k)\circ\cdots\circ \Phi_2(t_2)\circ\Phi_1(t_1)(p) \ | \ |t_i|<\epsilon, \ i=1,\dots,k \right\}\\ \subset \operatorname{B}(k\epsilon, p):= \left\{ q\in M \ | \ d(p,q)<k\epsilon \right\}.  
\end{align*}

We can move in the remaining directions along horizontal paths by successive applications of the commutators of flows, based on the bracket generating property of the distribution in the following way. Let $ \mathcal{H} $ denote also the sheaf of smooth vector fields tangent to $ \mathcal{H}, $ that is, the association between open sets $ U\subset M $ and horizontal smooth vector fields $ \mathcal{H}(U) $ defined locally on $ U $, and define the new vector sheaves
$$ \mathcal{H}^2 := \mathcal{H}\oplus [\mathcal{H},\mathcal{H}],\quad \mathcal{H}^{r+1} := \mathcal{H}^r\oplus [\mathcal{H}, \mathcal{H}^r], $$
where
$$ \left[ \mathcal{H}, \mathcal{H}^j \right]:= \operatorname{span} \left\{ [X,Y] \ | \ X\in \mathcal{H},\ Y \in \mathcal{H}^j \right\}, \text{ for }j=1,2,\dots  $$ 
The movement in the $ \mathcal{H}^2/\mathcal{H} $ directions can be made along horizontal paths by applying the commutators $ \Phi_{ij}(t):= [\Phi_i(t),\Phi_j(t)] $ to $ p. $ As $ \Phi_{ij}(t)(p)=p+t^2[X_i, X_j](p) $ (by lemma \ref{lm:approx}) if $ |t|\leq\epsilon, $ we will move by an amount $ \epsilon^2 $ in the $ \mathcal{H}^2/\mathcal{H} $ directions. This process is realized inductively until exhausting the tangent space. Let $ I=(i_1,i_2,\dots,i_n), $ $ 1\leq i_j\leq k $, and $ X_I:=[X_{i_1}, X_J], $ where $ J=(i_2,\dots,i_n), $  so that $ X_I $ is the iteration of $ n $ Lie brackets. Similarly, define flows $ \Phi_I(t):=[\Phi_{i_1}(t), \Phi_J(t)] $. By successive application of lemma \ref{lm:approx}, we have that   
$$ \Phi_I(t)= \mathbb{1} +t^n X_I+ O(t^{n+1}).$$ 
Due to the bracket generating condition over any local frame of the distribution, we can select a local frame for the entire tangent bundle amongst the $ X_I, $ letting $ n $ and $ i_1,\dots,i_n $ vary accordingly. We choose such a frame and relabel it as $ Y_i, $ $ i=1,\dots,m, $ in a way that it satisfies: $ Y_1=X_1,\dots, Y_k=X_k $ span $ \mathcal{H} $ near $ p, $ $ \left\{ Y_1,\dots,Y_{n_2} \right\} $ span $ \mathcal{H}^2 $ near $ p, $ $ \left\{ Y_1,\dots,Y_{n_2},\dots,Y_{n_3} \right\} $ span $ \mathcal{H}^3 $ near $ p $,  and so on. The tuple $ (k,n_2,n_3,\dots, n_r) $ is called the \textbf{growth vector of the distribution at $ p $}, and the smallest integer $ r $ is called the \textbf{step or degree of nonholonomy of the distribution at $ p. $ }   

\begin{df}[\cite{montgomery2002tour}, page 49.] Let $ \left\{ Y_1,\dots, Y_m \right\} $ be the local frame of the tangent bundle in a neighborhood of $ p\in M $ constructed as stated before using the local frame of a bracket generating distribution. Let $ Y_i=X_I $, $ I=(i_1,\dots,i_{n_I}) $. We denote the length of $ I $ as $ w_i:= \left| I \right| $, and define the assignment $ i\mapsto w_i, $ called the \textbf{weighting associated to the growth vector}. 
	
\end{df}

We relabel the flows $ \Phi_I $ in a similar fashion, so that $ \Phi_i $ is the flow associated to $ Y_i. $ In this case, each point $ \Phi_i(t)(p) $ is the endpoint of the concatenation of $ w_i $ simple horizontal curves, each one of length $ t $. If we impose that $ \left| t \right|\leq \epsilon, $ then $ \Phi_i(t)(p) $ lies in the ball of radius $ w_i\epsilon $ centered at $ p. $ Moreover, in coordinates we have that 
$$ \Phi_i(t)(p)=p+t^{w_i}X_i(p)+ O(t^{w_i+1}), $$ 
so $ \Phi_i(t)(p) $ lies in the Euclidean box of volume $ \epsilon^{w_i} $, in the $ \mathcal{H}^{w_i} $ directions. From this we can deduct that the subriemannian ball $ B(\epsilon,p) $ contains an Euclidean coordinate box whose sides are of order $ \epsilon^{w_i} $ in the $ i $-th direction. This result is known as the ball-box theorem, and it will be stated and proved after the following definitions.

\begin{df}[\cite{montgomery2002tour}, definition 2.4.1] 
	Coordinates $ y_1,\dots,y_m $ are said to be \textbf{linearly adapted to the distribution $\mathcal{H}$ at $ p $ } if $ \mathcal{H}^i(p):= \{ X(p) \ | \ X$ is a section of $\mathcal{H}$ defined on $p$$\}$ is annihilated by the differentials $ dy_{n_i+1},\dots, dy_m $ at $ p, $ where $ n_i=n_i(p) $ are the coordinates of the growth vector at $ p. $  
\end{df}
\begin{df}
	The $ w $- weighted box of size $ \epsilon $ is the point set
	$$ \operatorname{Box}^w(\epsilon) := \left\{ y\in \mathbb{R}^m \ | \ |y_i|\leq \epsilon^{w_i}; \ i=1,\dots,m \right\}.  $$ 
\end{df}

Set $ y_i $ as the coordinates centered at $ p $ such that $ dy_i(p) $ are the dual basis to the $ Y_i(p), $ so that these coordinates are linearly adapted to the distribution by definition.

\begin{thm}[Ball-Box Theorem]
	There exist linearly adapted coordinates $ y_1,\dots,y_m $ and positive constants $ c<C, \ \epsilon_0>0, $ such that for all $ \epsilon<\epsilon_0, $ 
	$$ \operatorname{Box}^w(c\epsilon)\subset B(\epsilon, p)\subset \operatorname{Box}^w(C\epsilon).    $$ 
\end{thm}
\begin{figure}[ht]
    \centering
    \incfig{ballbox}
    \caption{Ball box theorem.}
    \label{fig:ballbox}
\end{figure}
\begin{proof}[Proof of $\operatorname{Box}^w(c\epsilon)\subset B(\epsilon, p)$] (\cite{montgomery2002tour}, theorem 2.4.2). As the approximation $ \Phi_{ij}(t)= \mathbb{1} + t^2 \left[ X_i,X_j \right] + O(t^3)$ has the coefficient $ t^2 $ that is always positive, we can not move in the negative $ [X_i,X_j] $ direction. To circunvent this, we define 
	$$ \Psi_{i j}(t)=\left\{\begin{array}{ll}{\left[\Phi_{i}(t), \Phi_{j}(t)\right]} & t \geq 0 \\ {\left[\Phi_{j}(t), \Phi_{i}(t)\right]} & t<0,\end{array}\right. $$ 
	so that
	$$\Psi_{i j}(t)=\left\{\begin{array}{ll}I+t^{2} X_{i j}+O\left(t^{3}\right) & t \geq 0 \\ I-t^{2} X_{i j}+O\left(t^{3}\right) & t<0.\end{array}\right.$$
This problem occurs wheneve the number $ w_i $ associated to the flow $ \Phi_I $ of $ Y_i $ is even. To solve this in general, we set
$$\Psi_{I}(t)=\left\{\begin{array}{ll}
\Phi_{I}(t) & t \geq 0 \\
{\left[\Phi_{J}(t), \Phi_{i_{1}}(t)\right]} & t<0.
\end{array}\right.$$
If $ w_i $ is odd, we keep $ \Psi_I=\Phi_I, $ and we relabel $ \Psi_I $ as $ \Psi_i $ accordingly. Now, we introduce the functions
$$\sigma_{i}(t)=\left\{\begin{array}{ll}
t^{w_{i}} & w_{i} \text { even, } t \geq 0 \\
-t^{w_{i}} & w_{i} \text { even, } t<0 \\
t^{w_{i}} & w_{i} \text { odd }
\end{array}\right.
$$
to simplify the notation of the approximation
$$ \Psi_I(t)= \mathbb{1}+ \sigma_i(t) Y_i + O(t^{w_i+1}). $$ 
Define the map $ F: \mathbb{R}^m \rightarrow {M} $ as $ F(t_1,\dots,t_m) = \left( \Psi_m(t_m) \circ \cdots \circ \Psi_1(t_1)\right)(p), $ and let $ y_i $ be linearly adapted coordinates for which the $ dy_i $ are dual to the $ Y_i $ of the basis of the tangent space at the base point $ p. $ Then, in coordinates,

$$ F_i(t_1,\dots,t_m) = \sigma_i(t_i) + o(t_i^{w_i}) = \pm t_i^{w_i} + o(t_i^{w_i}), $$ 
where the error $ o(t_i^{w_i}) $ corresponds to the composition of the previous $ \Psi_1(t), \dots,\Psi_{i-1}(t) $ flows evaluated in the big $ O- $errors of the previous sequential arguments. It remains to understand and bound these error terms, and for that we introduce the new variables $ s_i=\sigma_i(t), $ so that $ F_i(t_1,\dots,t_m)= s_i + o(s_i). $  

The map $ \sigma=(\sigma_1,\dots,\sigma_m) $ is a homeomorphism between neighborhoods of $ 0 $ in $ \mathbb{R}^m, $ so let $ S $ be the inverse of $ \sigma $ that is given by $ S=(s_1,\dots,s_m), $ where $ s_i(t_i)=\pm |t_i|^{1/w_i}. $ Then, it follows that
$$ y\circ F\circ S (s_1,\dots,s_m) = (s_1,\dots,s_m) + o(|s|), $$ 
from where we deduce that $ F\circ S$ is differentiable at the origin because of the differentiability of both $ y $ and $ y\circ G\circ S $, and its derivative is the identity in the $ s-y $ coordinates, so it is also $ C^1 $ near $ 0. $ We will proceed applying the inverse function theorem to invert $ F\circ S $ in a neighborhood of $ p. $ 

Write $ F\circ S(s_1,\dots,s_m) $ in coordinates as $ (y_1,\dots,y_m).$ Then, there exists a constant $ \epsilon_0 $ and constants $ c=c(\epsilon_0), \ C=C(\epsilon_0) $ such that $ c|s_i| \leq |y_i(s_1,\dots,s_m)|  \leq C|s_i|$ whenever $ |s_i|\leq \epsilon_0. $ Apply the constraint $ |s_i|\leq \epsilon^{w_i} $ to the $ s_i. $ It is clear that the inverse image of the point set $ |s_i|\leq \epsilon^{w_i} $ under the map $ S $ is the $ \epsilon- $cube box $ \operatorname{Box}(\epsilon)= \left\{ (t_1,\dots,t_m)\ | \ |t_i|\leq \epsilon \right\},  $  so that applying $ F $ to the $ \epsilon- $cube gives
$$ \operatorname{Box}^{w}(c\epsilon) \subset F( \operatorname{Box}(\epsilon) ) \subset \operatorname{Box}^w(C\epsilon).   $$ 
Each curve in $ F( \operatorname{Box} (\epsilon) )$ is the endpoint of a horizontal curve starting at $ p $ whose length is less than $ M\epsilon $, where $ M=M(w) $ counts the number of concatenations involved in $ F. $ Consequently,
$$ F( \operatorname{Box} (\epsilon)) \subset B(M\epsilon, p), $$ 
from where we get that $ \operatorname{Box}^w (c\epsilon)\subset B(\epsilon,p). $ 
\end{proof}
The remaining containment in the Ball-Box theorem will not be proved in this text, as the proof of Chow's theorem only requires the first containment. The proof of the remaining containment can be found in \cite{montgomery2002tour}, section 2.7.

With the Ball-Box theorem, Chow's theorem can be derived easily.

\begin{thm}[Chow's Theorem]
	If $ \mathcal{H} $ is a bracket generating distribution on a connected manifold $ M $, then any two points of $ M $ can be connected by a horizontal path.
\end{thm}
\begin{proof} (\cite{montgomery2002tour}, page 52). The containment $ \operatorname{Box}^w(c\epsilon) \subset B(\epsilon,p)$ shows that $ B(\epsilon, p) $ is a neighborhood of $ p $, which means that the accesible points from $ p $ through horizontal curves form a neighborhood of $ p. $ Let $ q\in M $ be another arbitrary point on the connected manifold, and take a smooth curve $ \gamma $ connecting $ p $ and $ q. $  

	The image of $ \gamma $ is compact, so we can cover it with finitely many succesive box neighborhoods, where the Ball-Box theorem holds. Denote these neighborhoods as $ U_1,\dots,U_n, $ with each one of these boxes centered on succesive points $ p_i $ along the image of $ \gamma, $ and $ p_1=p, $ $ p_n=q. $ Moreover, these open sets can be chosen in orden to satisfy $ U_i\cap U_{i+1}\neq\emptyset, $ so that there exists $ q_i \in U_i\cap U_{i+1}. $ By the Ball-Box theorem, we have paths connecting $ p_i $ with $ q_i, $ and $ q_i $ with $ p_{i+1}. $ Concatenating these paths yields a piecewise horizontal path connecting $ p $ and $ q. $ 
	
\end{proof}
\begin{ex}[Cartan's Distribution]
	Consider the distribution over $ \mathbb{R}^3 $ determined by the 1-form $\omega =dy- zdx $, or in dual manner, by the smooth vector fields $ \partial /\partial x + z \partial / \partial y, \partial/\partial z. $ From its definition it is clear that Cartan's distribution is of rank 2, so we can not know beforehand if any two pair of points can be connected through a horizontal curve. Nevertheless, as 
	$$ \left[ \frac{\partial}{\partial z}, \frac{\partial}{\partial x}+z \frac{\partial}{\partial y}    \right] = \frac{\partial}{\partial y}  $$ 
and $ T \mathbb{R}^3 = \operatorname{span}\{  \partial/\partial x + z \partial / \partial y, \partial/\partial z, \partial/\partial y\} $, Cartan's distribution is bracket generating, so by Chow's theorem any two points in $ \mathbb{R}^3 $ can be connected by a horizontal path. 
\end{ex}

\section{Subriemannian metrics on Bundles}%
\label{sec:metrics_on_bundles}
\subsection{Ehresmann connection}%
\label{sub:ehresmann_connection}


Let $ \pi : Q \rightarrow {M} $ be a submersion, which means that for all $ q\in Q $, $ d\pi_q : T_q Q \rightarrow {T_{\pi(q)}}M $ is a surjective map. In the other hand, The fiber $ Q_m = \pi^{-1}(m) $, with $ m=\pi(q) $ is a submanifold because $ m $ is a regular value of $ \pi. $  

\begin{df}
	Given a submersion $ \pi:Q \rightarrow {M}, $ the \textbf{vertical space at $ q\in Q $}, denoted as $ V_q $, is the tangent space to the fiber $ Q_m, $ with $ m=\pi(q), $ that is,  
	$$ V_q = \operatorname{Ker}(d\pi_q) = T_q Q_m.  $$ 
\end{df}

The collection of vertical spaces is a distribution $ V \subset TQ $ that assigns to each $ q\in Q $ the space $ V_q. $ The distribution is by construction integrable, and its leaves are the fibers $ Q_m, $ $ m\in \pi(Q). $ 

\begin{df}[\cite{montgomery2002tour}, definition 11.1.1]
	An \textbf{Ehresmann connection} or simply \textbf{connection for a submersion} $ \pi:Q \rightarrow {M} $ is a distribution that is everywhere transverse to the vertical, that is,  
	$$ V_q \oplus \mathcal{H}_q = T_q Q,\ \text{for all } q\in Q, $$ 
	and is of complementary dimension to the vertical.
\end{df}

The restriction of $ d\pi_q $ to $ \mathcal{H}_q $ is a linear isomorphism because $ d\pi_q $ is surjective, and $ T_q Q / \operatorname{Ker}(d\pi_q) = T_q Q / V_q \cong \mathcal{H}_q \cong T_m M. $ If $ M $ is endowed with a metric, we can use this linear isomorphism to pull the inner product back to the horizontal distribution $ \mathcal{H}. $ This construction varies smoothly with $ q\in Q $ (\cite{montgomery2002tour}, page 194). With this, we obtain a subriemannian metric whose underlying distribution is the Ehresmann connection.

\begin{df}
	The subriemannian structure on $ Q $ induced by the Ehresmann connection $ \mathcal{H} $ for the submersion $ \pi:Q \rightarrow {M} $, the linear isomorphism $ \mathcal{H}_q \cong T_m M $ and the riemannian metric over $ M $ is called the \textbf{induced subriemannian structure}. 
\end{df}

\begin{df}
	The \textbf{horizontal lift} of a curve $ c: I \rightarrow {M} $ starting at $ m\in M $ is defined (if exists) as the unique curve $ \gamma : I \rightarrow {Q} $ that is tangent to $ \mathcal{H}, $ starts at $ q\in Q_m, $ and projects to $c$, that is, $ \pi\circ\gamma = c. $ 
\end{df}

\begin{rem}
	The connection $ \mathcal{H} $ is called complete if for every smooth curve $ c: I \rightarrow {M} $ has a horizontal lift.
\end{rem}

\begin{prop}
	The induced subriemannian structure satisfies the following properties:
	\begin{enumerate}
		\item The subriemannian length of a horizontal path on $ Q $ equals the riemannian length of its projection to $ M. $ 
		\item The horizontal lift of a riemannian geodesic in $ M $ is a subriemannian geodesic in $ Q. $ 
		\item The projection $ \pi $ is distance decreasing, that is, $ d_M(\pi(q_1), \pi(q_2))\leq d_Q(q_1,q_2), $ for all $ q_1,q_2\in Q. $ 
	\end{enumerate}
\end{prop}
\begin{proof}
	\begin{enumerate}
		\item Let $ \gamma: I \rightarrow {Q} $ be a horizontal path to $ \mathcal{H} $ starting at $ q\in Q. $ Then, its longitude is 

		$$\ell_Q(\gamma) = \int_0^1 ||\dot{\gamma} (t) ||_{Q} dt = \int_0^1 || d\pi_{\gamma(t)}(\dot{\gamma}(t))||_M dt,$$ 

		having in mind that $d\pi_{\gamma(t)}$ induces an isomorphism between $ \mathcal{H}_{\gamma(t)} $ and $ T_{\pi(\gamma(t))} M,$ and this isomorphism determines the induced subriemannian structure over $ Q. $ As $c=\pi\circ\gamma$ and $ d/dt(\pi\circ\gamma) = d\pi_{\gamma(t)}(\dot{\gamma}(t)) $, $ \ell_Q(\gamma)=\ell_M(c). $  

		\item Suppose that $ c: I \rightarrow {M} $ is a riemannian geodesic, and let $ \gamma: I \rightarrow {Q} $ be its respective horizontal lift. Take any other horizontal curve $ \eta: I \rightarrow {Q} $ with the same endpoints than $ \gamma. $ Then, by the first affirmation of this proposition and the fact that $ c $ is a geodesic, we have
			$$ \ell_Q(\eta) = \ell_M(\pi\circ \eta) \geq \ell_M(c) = \ell_Q(\gamma),  $$ 
		from where we get that $ \gamma: I \rightarrow {Q} $ is a subriemannian geodesic.

	\item Let $ q_1,q_2\in Q. $ If $ d_Q(q_1,q_2) = \infty $ the result is immediate. Else, let $ \left\{ \gamma_i \right\}_{i\in \mathbb{N}} $ be a minimizing sequence of $ d_Q(q_1,q_2). $ Now consider the sequence of projected curves $ \left\{ \pi\circ\gamma_i \right\}_{i\in \mathbb{N}}, $ from where there are two possibilities: if this last sequence is minimizing, $ d_Q(q_1, q_2) = d_M(\pi(q_1), \pi(q_2)),$ as $ \ell_Q(\gamma_i)=\ell_M(\pi\circ\gamma) $;  if it is not minimizing,  $d_Q(q_1,q_2) = \lim_{i \rightarrow \infty} \ell_Q(\gamma_i) = \lim_{i \rightarrow \infty} \ell_M(\pi\circ\gamma_i) > d_M(\pi(q_1),\pi(q_2))$ by definition. Therefore, $ d_Q(q_1,q_2)\geq d_M(\pi(q_1),\pi(q_2)).$
	\end{enumerate}
\end{proof}

\begin{df}
	A riemannian metric on $ Q $ is said to be \textbf{compatible} with the induced subriemannian metric if the algebraic splitting $ TQ = V \oplus \mathcal{H} $ is an orthogonal decomposition with respect to the riemannian metric on $ Q$.  
\end{df}

Thanks to the orthogonal algebraic splitting, a compatible riemannian metric is determined by its restriction to $ V, $ and the induced subriemannian structure restricted to $ \mathcal{H} $. This fact can be turned around in the following sense. Take a riemannian metric $ d^2S_Q $ on $ Q. $ Let $ V_q^\perp $ be the orthogonal complement of the vertical space $ V_q $ in $ q\in Q $ with respect to the riemannian metric, and set $ \mathcal{H}_q=V_q^\perp. $ The riemannian metric defines inner products in $ V_q^\perp $  and $ T_mM, $ being defined in the first as the restriction of the riemannian metric, and in the second is given by the metric and the isomorphism $ \mathcal{H}_q\cong T_mM. $ Nevertheless, this inner product on $ T_mM $ can depend on the point $ q\in Q_m $, as the metric could vary with the point of the fiber. If that is the case, we cannot construct a riemannian metric on $ M, $ as there is no consistent way to choose which $ q\in Q_m $ is going to be used to define the inner product (\cite{montgomery2002tour}, page 195).

\begin{df}
	Given a riemannian manifold $ (Q, d^2S_Q), \pi: Q \rightarrow {M}
	$ a submersion, we say that the pair $ (\pi, d^2S_Q) $ is a \textbf{riemannian submersion} if the inner product defined on $ T_mM \cong \mathcal{H}_q $ does not depend of the point $ q\in Q_m $ chosen. 
\end{df}

Equivalently, we can say that $ (\pi, d^2S_Q) $ is a riemannian submersion if for each $ q\in Q, $ the restriction of $ d\pi_q $ to the horizontal $ \mathcal{H}_q:=V_q^\perp $ is a linear isometry between inner product spaces.

\subsection{Metrics on Principal Bundles}%
\label{sub:metrics_on_principal_bundles}

\begin{df}[\cite{lee2003introduction}, page 268]
Let $ M, G, Q $ be manifolds. A \textbf{fiber bundle} over $ M $ with fiber $ G $ is a manifold $ Q, $ together with a smooth surjective map $ \pi: Q \rightarrow {M}
$ with the property that for each $ x\in M $ there is a neighborhood $ U $ of $ x $ and a diffeomorphism $ \phi:\pi^{-1}(U) \rightarrow {U\times G,}
 $ called local trivialization of $ Q $ over $ U, $ such that the following diagram commutes:
 \begin{center}
 	\begin{tikzcd}
\pi^{-1}(U) \arrow[rd, "\pi"'] \arrow[rr, "\phi"] &   & U\times G \arrow[ld, "\operatorname{proj}_U"] \\
                                                  & U &                                              
\end{tikzcd}
 \end{center}
 $ Q $ is called \textbf{total space of the bundle}, $ M $ is the \textbf{base} and $ \pi $ the \textbf{projection}.   
\end{df}

\begin{df}[\cite{montgomery2002tour}, page 196]
	The submersion $ \pi:Q \rightarrow {M}
	$ with Ehresmann connection $ \mathcal{H} $ is called a \textbf{principal $ G- $bundle with connection} if it is a fiber bundle whose fiber $ G $ is a Lie group ($G$ is a manifold with group structure), and this group acts on $ Q $ in such a way that the following properties hold:
	\begin{itemize}
		\item $ G $ acts freely on $ Q, $ which means that there is an action $ \alpha: Q\times g \rightarrow {Q,}
			$ denoted by $ \alpha(q,g) = q \cdot g, $ such that if $ q\cdot g=q, $ then $ g=e $ (the identity element of $ G $). 
		\item The group orbits are the fibers of $ \pi:Q \rightarrow {M}
			$, that is, for all $ m\in M, $ $ \pi^{-1}(m) = G_q := \left\{ q\cdot g \ | \ g\in G \right\}, $ with $ q\in Q_m. $ From this we can conclude that $ M $ is diffeomorphic to the quotient $ Q/G $ (equivalence classes of the relation $ p\sim q  $ if there is $ g\in G $ such that $ p=q\cdot g $). This diffeomorphism $ \phi: M \rightarrow {Q/G}
			$  is given by $ \phi(m)=[q], $ with $ \pi(q)=m. $   
		\item The $ G- $action on $ Q $ preserves the horizontal distribution $ \mathcal{H} $ in the sense that $ \mathcal{H}_{q\cdot g} = d(\alpha_g)_q (\mathcal{H}_q), $ for all $ q\in Q, $ $ g\in G $, where $ \alpha_g: Q \rightarrow {Q}
		 : q\mapsto q\cdot g.$  
	\end{itemize}
\end{df}

\begin{df}[\cite{montgomery2002tour}, definition 11.2.1]
	A subriemannian metric $ (\mathcal{H}, <\cdot,\cdot>) $ on the principal $ G- $bundle $ \pi: Q \rightarrow {M}
	 $ is called a \textbf{metric of bundle type} if the inner product $ <\cdot,\cdot> $ on $ \mathcal{H}$is induced from a riemannian metric on $ M. $ 
\end{df}
\bibliographystyle{unsrt}
\nocite{*}
\bibliography{mybibliography}

\end{document}
