\documentclass [xcolor=svgnames, t] {beamer} 
\usepackage[utf8]{inputenc}
\usepackage{booktabs, comment} 
\usepackage[absolute, overlay]{textpos} 
\usepackage{pgfpages}
\usepackage[font=footnotesize]{caption}
\useoutertheme{infolines} 

\definecolor{gold}{RGB}{254, 206, 0}

\setbeamercolor{title in head/foot}{bg=gold, fg=black}
\setbeamercolor{author in head/foot}{bg=myuniversity}
\setbeamertemplate{page number in head/foot}{}
\usepackage{csquotes}


\usepackage{amsmath}
\usepackage[makeroom]{cancel}
\usepackage{tikz-cd}
\usepackage{ragged2e}
\renewcommand{\raggedright}{\leftskip=0pt \rightskip=0pt plus 0cm}
\usepackage{textpos}
\apptocmd{\frame}{}{\justifying}{} % Allow optional arguments after frame.
\usetheme{Madrid}
\definecolor{myuniversity}{RGB}{0, 85, 150}
\usecolortheme[named=myuniversity]{structure}

%%%%%%%% THEOREMS %%%%%%%%%%%%
\theoremstyle{definition}
%\newtheorem{df}{Definition}
\newtheorem{df}{Definition}
\theoremstyle{plain}
\newtheorem{prop}{Proposition}
%\newtheorem{prop}{Proposición}
\newtheorem{thm}{Theorem}
%\newtheorem{thm}{Teorema}
\newtheorem{lm}{Lemma}
%\newtheorem{lm}{Lema}
\newtheorem{cor}{Corollary}
%\newtheorem{cor}{Corolario}
\theoremstyle{remark}
\newtheorem{ex}{Example}
%\newtheorem{ex}{Ejemplo}
\newtheorem{rem}{Remark}
%\newtheorem{rem}{Observación}


\title[Sub-Riemannian Geometry]{Elements of Sub-Riemannian Geometry and its Applications}
\institute[]{Departamento de Matemáticas \\ Universidad de los Andes}
\titlegraphic{\includegraphics[height=2.5cm]{Uandes.jpg}}
\author[Julián Jiménez Cárdenas]{
Julián Jiménez Cárdenas}


\institute[]{Departamento de Matemáticas \\ Universidad de los Andes}
\date{June 01, 2022}


\addtobeamertemplate{navigation symbols}{}{%
    \usebeamerfont{footline}%
    \usebeamercolor[fg]{footline}%
    \hspace{1em}%
    \insertframenumber/\inserttotalframenumber
}

\begin{document}
\begin{frame}
\maketitle
\end{frame}


%%%%%%%%%%%%%%%%%%%%%%%%%%%%
%\logo{\includegraphics[scale=0.15]{Uandes.jpg}~%
%}


%%%%%%%%%%%%%%%%%%%%%%%%%%



\begin{frame}
\frametitle{Table of Contents}
\tableofcontents
\end{frame}

\section{Distributions}
\subsection{Distributions and horizontal curves}%
\label{sub:distributions_and_horizontal_curves}


\begin{frame}[fragile]
	\begin{block}{Vector Bundle}
	Let $ E,M $ be two smooth manifolds. A map $ \pi: E \rightarrow M $ is said to be a \textit{smooth vector bundle} if it is a submersion, its fibers have the structure of finite dimensional vector spaces, and of every $ p\in M $ there is an open neighborhood $ U$ of $ p $  and a diffeomorphism $ \phi: U \times \mathbb{R}^k \rightarrow \pi^{-1}(U) $ such that the following diagram commutes 	
	\begin{center}
	% https://tikzcd.yichuanshen.de/#N4Igdg9gJgpgziAXAbVABwnAlgFyxMJZABgBpiBdUkANwEMAbAVxiRAFUAdTvAW3gAE3XnRwALAEYTgAJQC+APQDWIOaXSZc+QigBM5KrUYs23NFgXAAtAEY5ACnYBKVepAZseAkRukbh+mZWRA5VQxgoAHN4IlAAMwAnCF4kMhAcCCR9IyDTTgg0GATRCASwOn5gNCSAKzkAfXYQagY6CRgGAAVNLx0QBKxIsRxXeKSUxDSMpF8ckxCzMSxRkETkmeppxGzA+ZAzZbkKOSA
\begin{tikzcd}
U\times \mathbb{R}^k \arrow[rd, "\operatorname{proj}_U"'] \arrow[rr, "\phi"] &   & \pi^{-1}(U) \arrow[ld, "\pi"] \\
                                                                             & U &                              
\end{tikzcd}	
	\end{center}
	and the map $ v \mapsto \phi(p,v) $ is a linear isomorphism between $ \pi^{-1}(p) $ and $ \mathbb{R}^k. $ 
	\end{block}
	\begin{rem}
		
	\end{rem}
\end{frame}
\section{Literature Review}

\section{Methodology}

\section{Experiment}

\section{Conclusion}
    

\begin{frame} [allowframebreaks]\frametitle{References}
        \bibliographystyle{unsrt}
	\nocite{*}
        \bibliography{../mybibliography.bib}
\end{frame}

\end{document}
